\documentclass{beamer}
\mode<presentation>
\setbeamertemplate{bibliography item}{}
\usepackage[utf8]{vietnam}
\usepackage{beamerthemesplit}
\usepackage{graphicx}
\usepackage{booktabs}
\usepackage{amsmath}
\usepackage{textpos}
\usepackage{pgfplots}
\usepackage{tikz}
\usepackage{hyperref}
\usepackage{caption}
\usetikzlibrary{shapes.geometric, arrows}
\usetikzlibrary {datavisualization} 
\pgfplotsset{compat=1.18, width = 7cm}
\usetikzlibrary{patterns}
\setbeamertemplate{bibliography item}[text]
\usetheme{AnnArbor} % AnnArbor, Ilmenau, Darmstadt, Dresden, CambridgeUS, Frankfurt, Singapore
\newtheorem{dn}{Định nghĩa}[section]
\newtheorem{dl}{Định lý}[section]
\newtheorem{tc}{Tính chất}[section]
\newtheorem{hq}{Hệ quả}[section]
\newtheorem{bd}{Bổ đề}[section]
\newtheorem{md}{Mệnh đề}[section]
\newtheorem{vd}{Ví dụ}[section]
\newtheorem{nx}{Nhận xét}[section]
\newcommand{\dom}{\text{{\rm dom}}}
\newcommand{\epi}{\text{{\rm epi}}}
\newcommand{\Min}{\text{{\rm Min}}}
\setbeamertemplate{theorems}[numbered]
\setbeamertemplate{definitions}[numbered]
\setbeamertemplate{footline}[frame number]
\usepackage{algorithm}
\usepackage{color}
\usepackage{algorithmic}
\usepackage{footmisc}
\usepackage{indentfirst} 
\usepackage{comment}
\AtBeginEnvironment{proof}{%
  \setbeamercolor{block title}{use=example text,fg=white,bg=example text.fg!75!black}
  \setbeamercolor{block body}{parent=normal text,use=block title example,bg=block title example.bg!10!bg}
}
\renewcommand{\thefootnote}{\arabic{footnote}}
\usefonttheme{professionalfonts}
\setbeamercolor{normal text}{bg=white,fg=black}
\renewcommand{\thefootnote}{\arabic{footnote}}
\beamertemplatetransparentcoveredhigh
\begin{document}
\title[]{\fontsize{13pt}{10pt}\selectfont {\bf \LARGE   Phương pháp giải bài toán \\Tối ưu tuyến tính nguyên}\\
------------------------------------------

{\small Hướng dẫn: PGS.TS. Tạ Quang Sơn}} 
\author[]{\bf Thực hiện : ĐỖ NGỌC MINH THƯ \& NGUYỄN CHÍ BẰNG \\
Sinh viên lớp: DTU1221, Khóa: 22}
\institute[Báo cáo luận văn thạc sĩ]{\fontsize{2pt}{2pt}}%
\small{\date{\today}}
\begin{frame}
\begin{center}
{\fontsize{8pt}{8pt}\selectfont \bf{ỦY BAN NHÂN DÂN THÀNH PHỐ HỒ CHÍ MINH\\
TRƯỜNG ĐẠI HỌC SÀI GÒN}}
\end{center}
\begin{center}
\end{center}

\begin{center}
{\fontsize{10pt}{6pt}\selectfont \bf{BÁO CÁO ĐỀ CƯƠNG NGHIÊN CỨU KHOA HỌC\\
NGÀNH: TOÁN ỨNG DỤNG}}
\end{center}
\titlepage
\end{frame}
\begin{frame}
    \frametitle{NỘI DUNG BÁO CÁO}
    \tableofcontents
\end{frame}

% Tai sao quan tam de tai?
% Lam duoc gi?
% Hoc duoc gi ?
% Du dinh tiep theo ?
% - - -
% Cơ sở lý thuyết và Mục tiêu
\section{Giới thiệu}

\begin{frame}
   \center 
   \huge Giới thiệu
\end{frame}

\begin{frame}{\bf Mục đích nghiên cứu}
\end{frame}

\begin{frame}{\bf Bài toán Tối ưu tuyến tính}
\end{frame}


\section{Phương pháp lát cắt Gomory}
\section{Phương pháp Land-Doig (Nhánh cận)}
\section{Kết luận và Hướng phát triển}
\section{Tài liệu tham khảo}

%sample
\begin{frame}
\end{frame}

\begin{frame}
    \begin{block}{}
    \medskip
    \center{\huge \it \textcolor[rgb]{0.50,0.30,1.0}{Cảm ơn quý thầy cô và các anh chị đã quan tâm theo dõi!}}
    \medskip
    \end{block}	
\end{frame}    
\end{document}