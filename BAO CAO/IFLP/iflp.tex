\documentclass{beamer}
\mode<presentation>
\setbeamertemplate{bibliography item}{}
\usepackage[utf8]{vietnam}
\usepackage{beamerthemesplit}
\usepackage{graphicx}
\usepackage{booktabs}
\usepackage{amsmath}
\usepackage{textpos}
\usepackage{pgfplots}
\usepackage{tikz}
\usepackage{hyperref}
\usepackage{caption}
\usetikzlibrary{shapes.geometric, arrows}
\usetikzlibrary {datavisualization} 
\pgfplotsset{compat=1.18, width = 7cm}
\usetikzlibrary{patterns}
\setbeamertemplate{bibliography item}[text]
\usetheme{Ilmenau} % AnnArbor, Ilmenau, Darmstadt, Dresden, CambridgeUS, Frankfurt, Singapore
\newtheorem{dn}{Định nghĩa}[section]
\newtheorem{dl}{Định lý}[section]
\newtheorem{tc}{Tính chất}[section]
\newtheorem{hq}{Hệ quả}[section]
\newtheorem{bd}{Bổ đề}[section]
\newtheorem{md}{Mệnh đề}[section]
\newtheorem{vd}{Ví dụ}[section]
\newtheorem{nx}{Nhận xét}[section]
\newcommand{\dom}{\text{{\rm dom}}}
\newcommand{\epi}{\text{{\rm epi}}}
\newcommand{\Min}{\text{{\rm Min}}}
\setbeamertemplate{theorems}[numbered]
\setbeamertemplate{definitions}[numbered]
\setbeamertemplate{footline}[frame number]
\usepackage{algorithm}
\usepackage{color}
\usepackage{algorithmic}
\usepackage{footmisc}
\usepackage{indentfirst} 
\usepackage{comment}
\AtBeginEnvironment{proof}{%
  \setbeamercolor{block title}{use=example text,fg=white,bg=example text.fg!75!black}
  \setbeamercolor{block body}{parent=normal text,use=block title example,bg=block title example.bg!10!bg}
}
\renewcommand{\thefootnote}{\arabic{footnote}}
\usefonttheme{professionalfonts}
\setbeamercolor{normal text}{bg=white,fg=black}
\renewcommand{\thefootnote}{\arabic{footnote}}
\beamertemplatetransparentcoveredhigh
\title[]{\fontsize{13pt}{10pt}\selectfont {\bf \LARGE Tối ưu phân tuyến tính \\ cho nghiệm nguyên}\\}
\author[]{\bf Nguyễn Chí Bằng \\}
\small{\date{\today}}

\begin{document}

\begin{frame}
    \titlepage
\end{frame}

\begin{frame}{TÓM TẮT}
\Large
\begin{itemize}
\item Giới thiệu về bài toán tối ưu phân tuyến tính:
\begin{itemize} \Large
\item Cơ sở lý thuyết.
\item Thuật toán Dinkelbach.
\end{itemize}
\item Phương pháp giải bài toán tối ưu phân tuyến tính cho nghiệm nguyên bằng thuật toán nhánh cận (LandDoig).
\end{itemize}
\end{frame}

\begin{frame}
    \frametitle{NỘI DUNG}
    \tableofcontents
\end{frame}

\section{Giới thiệu}
%FLP-begin
\begin{frame}
   \center 
   \Huge Giới thiệu bài toán 
\end{frame}
\begin{frame}{Tối ưu phân tuyến tính (Linear-Fractional Programming)}
    \begin{equation} \small \label{F}
        \begin{split}
        (F) \quad Q(x) & = \frac{P(x)}{D(x)} \quad \longrightarrow Max \\
            & \left\{
            \begin{split}
            &Ax \leq  b, \\
            &x \geq 0. \\
            \end{split}
            \right.    
        \end{split}
    \end{equation}            
    \begin{itemize} \small
    \item Bài toán $(F)$ gọi là bài toán \textbf{Tối ưu phân tuyến tính.}
    \item Trong đó $A$ là ma trận $m\times n$, $b=\begin{pmatrix}
        b_1 \\
        b_2 \\
        \vdots \\
        b_m
        \end{pmatrix}$, với $x\in \mathbb{R}^n_+$. Tập $S_F:=\{x\in \mathbb{R}^n_+: Ax\leq b\}$ là tập nghiệm của bài toán Tối ưu phân tuyến tính. 
    \item $P(x)=p^Tx+p_0$, với $p^T = (p_1 \: p_2 \: \ldots \: p_n)$ và $D(x)=d^Tx+d_0$, với $d^T = (d_1 \: d_2 \: \ldots \: d_n)$ ($D(x)>0, \forall x \in S_F$).
    \end{itemize}
\end{frame}
\begin{frame}{Bài toán minh hoạ} \Large
    \begin{equation}
    \begin{split}
    \quad Q(x) & = \frac{4x_1+2x_2-6}{3x_1+2x_2-5} \quad \longrightarrow Max \\
        & \left\{
        \begin{split}
        & x_1 + x_2 \geq 6 \\
        & x_1 + 2x_2 \leq 12 \\
        &x_1, x_2 \geq 0. \\
        \end{split}\right.    
    \end{split}
    \end{equation}            
\end{frame}
\begin{frame}{Mối quan hệ với bài toán tối ưu tuyến tính}
\begin{itemize}
\item Nếu $d^T=0$ và $d_0=1$, bài toán (F) trở thành bài toán tối ưu tuyến tính (P) và ta gọi (F) là bài toán mở rộng của (P):
\begin{equation} \small \label{P}
    \begin{split}
    (P) \quad P(x) & = p^Tx+p_0 \quad \longrightarrow Max \\
        & \left\{
        \begin{split}
        &Ax \leq  b, \\
        &x \geq 0. \\
        \end{split}
        \right.    
    \end{split}
\end{equation}
\item Nếu $d^T=0$ và $d_0 \neq 0$, ta thu được bài toán tuyến tính (Q):
\begin{equation} \small \label{Q}
    \begin{split}
    (Q) \quad Q(x) & = \frac{p^T}{d_0}x+\frac{p_0}{d_0}=\frac{P(x)}{d_0} \quad \longrightarrow Max \\
        & \left\{
        \begin{split}
        &Ax \leq  b, \\
        &x \geq 0. \\
        \end{split}
        \right.    
    \end{split}
\end{equation}
\end{itemize}
\end{frame}
\begin{frame}
\begin{itemize}
\item Ngược lại nếu $p^T=0$ và $p_0 \neq 0$:
\begin{equation} \small \label{Q}
    \begin{split}
    (Q) \quad Q(x) & = \frac{p_0}{d^Tx+d_0} = \frac{p_0}{D(x)} \quad \longrightarrow Max \\
        & \left\{
        \begin{split}
        &Ax \leq  b, \\
        &x \geq 0. \\
        \end{split}
        \right.    
    \end{split}
\end{equation}
Tương tự bài toán:
\begin{equation} \small \label{Q}
    \begin{split}
    (Q) \quad Q(x) & = \frac{d^Tx+d_0}{p_0} = \frac{D(x)}{p_0} \quad \longrightarrow Min \\
        & \left\{
        \begin{split}
        &Ax \leq  b, \\
        &x \geq 0. \\
        \end{split}
        \right.    
    \end{split}
\end{equation}
\item Nếu $p^T$ và $d^T$ phụ thuộc tuyến tính, khi đó tồn tại $\mu \neq 0$ và $p^T=\mu d^T$, ta thu được hàm:
\end{itemize}
\end{frame}
\begin{frame} \large
\begin{equation} \label{Q}
    \begin{split}
    (Q) \quad Q(x) & = \frac{\mu d^Tx + p_0}{d^Tx+d_0} = \mu + \frac{p_0-\mu d_0}{d^Tx+d_0} \quad \\
        & \left\{
        \begin{split}
        &Ax \leq  b, \\
        &x \geq 0. \\
        \end{split}
        \right.    
    \end{split}
\end{equation}
Ta thay bằng hàm $D(x)$ với điều kiện:
\begin{itemize}
\item Nếu $p_0 - \mu d_0 > 0$, $D(x) \longrightarrow Min$.
\item Nếu $p_0 - \mu d_0 < 0$, $D(x) \longrightarrow Max$.
\item Nếu $p_0 - \mu d_0 = 0$ thì $Q(x)=\mu= \text{hằng số} \: (\forall x \in S_F)$, ta bỏ qua bài toán.
\end{itemize}
\end{frame}
%FLP-end

%GRAPHICAL-BEGIN
\section{Phương pháp hình học}
\begin{frame}
   \center 
   \Huge Phương pháp hình học 
\end{frame}

\begin{frame}{Bài toán trên không gian $\mathbb{R}^2$}
\begin{equation} \large
    \begin{split}
    (F) \quad Q(x) & = \frac{P(x)}{D(x)} = \frac{p_1x_1 + p_2 x_2 + p_0}{d_1x_1+d_2x_2 + d_0}\quad \longrightarrow Max \\
        & \left\{
        \begin{split}
        &Ax \leq  b, \text{ trong đó } A=m \times 2 \\
        &x_1, x_2 \geq 0. \\
        \end{split}
        \right.    
    \end{split}
\end{equation}
\end{frame}

\begin{frame}
\begin{figure}
    \begin{tikzpicture}[scale=1.2]
    \begin{axis}
        [
        xmin=0,xmax=80,
        ymin=0,ymax=80,
        xlabel={$x_1$},
        ylabel={$x_2$},
        grid style={line width=.1pt, draw=darkgray!50},
        major grid style={line width=.2pt,draw=darkgray!50},
        axis lines=middle,
        yticklabel=\empty,
        xticklabel=\empty,
        enlargelimits={abs=0},
        samples=100,
        domain = -20:20,
        ]
        \filldraw[blue!30, pattern=north west lines, pattern color=blue!30, line width=1pt] (30, 25) -- (50, 20) -- (70, 30) -- (70, 50) -- (50, 65) -- (20, 50) -- cycle;
        \node[blue!50] at (60,40) {\tiny $S_F$};
    \end{axis}
    \end{tikzpicture}  
    \caption{Tập nghiệm minh hoạ của bài toán (F)}
    \end{figure}
\end{frame}
\begin{frame}{Tính chất}
Đặt $Q(x)=K$ ($K \in \mathbb{R}$), ta được:
\begin{equation} \label{pt0}
(p_1-Kd_1)x_1+(p_2-Kd_2)x_2+(p_0-Kd_0) = 0
\end{equation}
\begin{equation} \label{pt1}
\implies
\left\{ \large
\begin{array}{rcr}
p_1x_1+p_2x_2+p_0 &=& 0 \\
d_1x_1+d_2x_2+d_0 &=& 0 \\
\end{array}
\right.
\end{equation}
\begin{itemize}
\item $Q(x)=K$ là đường mức quét qua tập $S_F$, đến khi gặp \textsl{cực điểm} thì ở đó ta nhận được giá trị $K$ là giá trị tối ưu của bài toán (F).
\item Ta xác định được điểm cố định $F$ là nghiệm của phương trình \eqref{pt0}, nói cách khác, điểm cố định $F$ là điểm giao của 2 đường thẳng $P(x)=0$ và $D(x)=0$.
\item Trường hợp $P(x)=0$ song song với $D(x)=0$, hay nói cách khác hệ \eqref{pt0} vô nghiệm thì điểm cố định $F$ không tồn tại.
\end{itemize}
\end{frame}
\begin{frame}
\begin{figure}
    \begin{tikzpicture}[scale=1.2]
    \begin{axis}
        [
        xmin=-10,xmax=80,
        ymin=-10,ymax=80,
        xlabel={$x_1$},
        ylabel={$x_2$},
        grid style={line width=.1pt, draw=darkgray!50},
        major grid style={line width=.2pt,draw=darkgray!50},
        axis lines=middle,
        yticklabel=\empty,
        xticklabel=\empty,
        enlargelimits={abs=0},
        samples=100,
        domain = -20:20,
        ]
        \filldraw[blue!30, pattern=north west lines, pattern color=blue!30, line width=1pt] (30, 25) -- (50, 20) -- (70, 30) -- (70, 50) -- (50, 65) -- (20, 50) -- cycle;
        \node[blue!50] at (60,40) {\tiny $S_F$};
        \node at (20, 25) {\small $F$};
        \node at (20,19.5) {\large \textbullet};
        \node at (55, -7) {\tiny $D(x)=0$};
        \node[rotate=60] at (42, 70) {\tiny $P(x)=0$};
        \draw (15, 10) -- (50, 80);
        \draw (10, 25) -- (70, -7);
    \end{axis}
    \end{tikzpicture}  
    \caption{Minh hoạ điểm cố định $F$}
    \end{figure}
\end{frame}
\begin{frame}
\begin{figure}
    \begin{tikzpicture}[scale=1.2]
    \begin{axis}
        [
        xmin=-10,xmax=80,
        ymin=-10,ymax=80,
        xlabel={$x_1$},
        ylabel={$x_2$},
        grid style={line width=.1pt, draw=darkgray!50},
        major grid style={line width=.2pt,draw=darkgray!50},
        axis lines=middle,
        yticklabel=\empty,
        xticklabel=\empty,
        enlargelimits={abs=0},
        samples=100,
        domain = -20:20,
        ]
        \filldraw[blue!30, pattern=north west lines, pattern color=blue!30, line width=1pt] (30, 25) -- (50, 20) -- (70, 30) -- (70, 50) -- (50, 65) -- (20, 50) -- cycle;
        \node[blue!50] at (60,40) {\tiny $S_F$};
        \node at (20, 25) {\small $F$};
        \node at (20,19.5) {\large \textbullet};
        \node at (55, -7) {\tiny $D(x)=0$};
        \node[rotate=60] at (42, 70) {\tiny $P(x)=0$};
        \draw (15, 10) -- (50, 80);
        \draw (10, 25) -- (70, -7);
        \draw[blue] (5, 19) -- (80, 20);
        \draw[blue] (20, 11) -- (20, 72);
        \draw [red] (6, 7) -- (77, 73);
        \node at (65, 15) {\tiny \color{blue} $Q(x)= Min$};
        \node[rotate=90] at (15, 60) {\tiny \color{blue} $Q(x)= Max$};
        \node[rotate=35] at (70, 60) {\tiny \color{red} $Q(x)=K$};
    \end{axis}
    \end{tikzpicture}  
    \caption{Minh hoạ đường mức $Q(x)=K$}
    \end{figure}
\end{frame}
\begin{frame}{Xét tính biến thiên} \large
Từ phương trình \eqref{pt0}, ta có:
\begin{equation}
x_2=-\frac{p_1-Kd_1}{p_2-Kd_2}x_1-\frac{p_0-Kd_0}{p_2-Kd_2}.
\end{equation}
Đạo hàm 2 vế ta được:
\begin{equation*}
\begin{split}
    \frac{dx_2}{dx_1} &= \frac{d}{dx_1} \Bigg[-\frac{p_1-Kd_1}{p_2-Kd_2}x_1\Bigg] - \frac{d}{dx_1} \Bigg[\frac{p_0-Kd_0}{p_2-Kd_2} \Bigg] \\
    \frac{dx_2}{dx_1} &= -\frac{p_1-Kd_1}{p_2-Kd_2} \\
    k &= -\frac{p_1-Kd_1}{p_2-Kd_2} \text{    (Đặt $k=\frac{dx_2}{dx_1}$)}
\end{split} 
\end{equation*}
\end{frame}

\begin{frame} \large
\begin{itemize}
\item Ta thấy hệ số góc $k$ phụ thuộc vào tham số $K$, Khảo sát sự biến thiên của $k$ theo $K$, ta có:
\begin{equation}
\frac{dk}{dK}=\frac{d_1p_2-d_2p_1}{(p_2-Kd_2)^2}
\end{equation}
\item Giá trị của $Q(x)$ tăng hay giảm phụ thuộc vào $(d_1p_2-d_2p_1)$, do đó $k$ biến thiên theo 1 chiều nhất định. Từ đây, ta có thể tìm được nghiệm tối ưu của bài toán (F).
\item Quay đường mức $Q(x)=K$ quanh điểm $F$ đến khi trùng với cực điểm $x^*$ ta nhận được giá trị cực đại của hàm $Q(x)$, $x^{**}$ ta nhận được giá trị cực tiểu của hàm $Q(x)$. 
\end{itemize}
\end{frame}

\begin{frame}
\begin{figure}
    \begin{tikzpicture}[scale=1.2]
    \begin{axis}
        [
        xmin=-10,xmax=80,
        ymin=-10,ymax=80,
        xlabel={$x_1$},
        ylabel={$x_2$},
        grid style={line width=.1pt, draw=darkgray!50},
        major grid style={line width=.2pt,draw=darkgray!50},
        axis lines=middle,
        yticklabel=\empty,
        xticklabel=\empty,
        enlargelimits={abs=0},
        samples=100,
        domain = -20:20,
        ]
        \filldraw[blue!30, pattern=north west lines, pattern color=blue!30, line width=1pt] (30, 25) -- (50, 20) -- (70, 30) -- (70, 50) -- (50, 65) -- (20, 50) -- cycle;
        \node[blue!50] at (60,40) {\tiny $S_F$};
        \node at (20, 25) {\small $F$};
        \node at (20,19.5) {\large \textbullet};
        \node at (55, -7) {\tiny $D(x)=0$};
        \node[rotate=60] at (42, 70) {\tiny $P(x)=0$};
        \draw (15, 10) -- (50, 80);
        \draw (10, 25) -- (70, -7);
        \draw[red] (5, 19) -- (80, 20);
        \draw[blue] (20, 11) -- (20, 72);
        \node at (60, 15) {\tiny \color{red} $Q(x)= Min = K$};
        \node[rotate=90] at (15, 60) {\tiny \color{blue} $Q(x)= Max$};
        \node[blue] at (53, 27) {\small $x^{**}$};
        \node[blue] at (50,20) {\large \textbullet};
    \end{axis}
    \end{tikzpicture}  
    \caption{$Q(x)$ đạt giá trị tối ưu lại điểm $x^{**}$}
    \end{figure}
\end{frame}
\begin{frame}
\begin{figure}
    \begin{tikzpicture}[scale=1.2]
    \begin{axis}
        [
        xmin=-10,xmax=80,
        ymin=-10,ymax=80,
        xlabel={$x_1$},
        ylabel={$x_2$},
        grid style={line width=.1pt, draw=darkgray!50},
        major grid style={line width=.2pt,draw=darkgray!50},
        axis lines=middle,
        yticklabel=\empty,
        xticklabel=\empty,
        enlargelimits={abs=0},
        samples=100,
        domain = -20:20,
        ]
        \filldraw[blue!30, pattern=north west lines, pattern color=blue!30, line width=1pt] (30, 25) -- (50, 20) -- (70, 30) -- (70, 50) -- (50, 65) -- (20, 50) -- cycle;
        \node[blue!50] at (60,40) {\tiny $S_F$};
        \node at (20, 25) {\small $F$};
        \node at (20,19.5) {\large \textbullet};
        \node at (55, -7) {\tiny $D(x)=0$};
        \node[rotate=60] at (42, 70) {\tiny $P(x)=0$};
        \draw (15, 10) -- (50, 80);
        \draw (10, 25) -- (70, -7);
        \draw[blue] (5, 19) -- (80, 20);
        \draw[red] (20, 11) -- (20, 72);
        \node at (65, 15) {\tiny \color{blue} $Q(x)= Min$};
        \node[rotate=90] at (15, 60) {\tiny \color{red} $Q(x)= Max=K$};
        \node[blue] at (25, 50) {\small $x^{*}$};
        \node[blue] at (20,50) {\large \textbullet};
    \end{axis}
    \end{tikzpicture}  
    \caption{$Q(x)$ đạt giá trị tối ưu lại điểm $x^*$}
    \end{figure}
\end{frame}

\begin{frame}{Ví dụ minh hoạ}
\begin{equation} \large \label{F}
    \begin{split}
    (F) \quad Q(x) & = \frac{6x_1+3x_2+6}{5x_1+2x_2+5} \quad \longrightarrow Max \\
        & \left\{
        \begin{split}
        &4x_1 - 2x_2 \leq  20, \\
        &3x_1+5x_2 \leq 25 \\
        &x_1, x_2 \geq 0. \\
        \end{split}
        \right.    
    \end{split}
\end{equation}            
\end{frame}

\begin{frame}
    \begin{center}
        \begin{figure}
        \begin{tikzpicture} [scale=1.3]
        \begin{axis}
            [
            xmin=-3,xmax=10,
            ymin=-3,ymax=7,
            grid=both,
            grid style={line width=.1pt, draw=darkgray!50},
            major grid style={line width=.2pt,draw=darkgray!50},
            axis lines=middle,
            minor tick num=1,
            enlargelimits={abs=0},
            samples=100,
            domain = -20:20,
            ]
            \filldraw[green, pattern=north west lines, pattern color=green, line width=2pt] (0, 0) -- (0, 5) -- (6, 1.4) -- (5, 0) -- cycle;
            \draw (-1.2, -1) -- (0.2, 6);
            \draw (-1.9, -0.6) -- (5.4, 4);
            \draw (-1,5.6) -- (9.6,-0.7) ;
            \draw (4.2,-1) -- (6.7, 2.4);
            \draw (-2,0) -- (9,0);
            \node at (-1.5, 0.5) {$F$};
            \node at (1, 5.5) {$A$};
            \node at (5.5, -0.5) {$B$};
            \node at (0.5, -0.5) {$C$};
        \end{axis}
        \end{tikzpicture}  
        \caption{Tập nghiệm của bài toán}
        \end{figure}
        \end{center}      
\end{frame}

\begin{frame}
Ta có hệ:
\begin{equation*}
    \begin{split}
        & \left\{
        \begin{split}
        &6x_1 - 3x_2 = -6, \\
        &5x_1+2x_2 = -5 \\
        \end{split}
        \right.    
    \end{split}
\end{equation*}
$\Rightarrow$ điểm cố định $F=(-1,0)$
Ta có các cực điểm 
\begin{equation*}
A=(0,5), B=(5,0), C=(0,0)
\end{equation*}
với giá trị
\begin{equation*}
Q(A)=\frac{21}{15}, Q(B)-\frac{18}{15}, Q(C)=\frac{18}{15}
\end{equation*}
Hàm $Q(x)$ đạt cực đại tại $A=(0,5)$.
\end{frame}
%GRAPHICAL-END
%DINKELBACH-start
\section{Thuật toán Dinkelbach}
\begin{frame}
   \center 
   \Huge Thuật toán Dinkelbach
\end{frame}

\begin{frame}{Tính chất} \large
\begin{itemize}
\item Quay lại bài toán:
\begin{equation} \label{F}
    \begin{split}
    (F) \quad Q(x) & = \frac{P(x)}{D(x)} \quad \longrightarrow Max \\
        & \left\{
        \begin{split}
        &Ax \leq  b, \\
        &x \geq 0. \\
        \end{split}
        \right.    
    \end{split}
\end{equation}            
\item Ta đặt hàm:
\begin{equation}
F(\lambda) = \underset{x \in S_F}{\max}\{P(x)-\lambda D(x)\}, \: \lambda \in \mathbb{R}
\end{equation}
\end{itemize}
\end{frame}

\begin{frame}
\begin{dl}
Vector $x^*$ là nghiệm tối ưu của bài toán (F) nếu và chỉ nếu 
\begin{equation}
F(\lambda^*) = \underset{x \in S_F}{\max}\{P(x)-\lambda^* D(x)\} = 0 
\end{equation}
Với
\begin{equation}
\lambda^* = \frac{P(x^*)}{D(x^*)}
\end{equation}
\end{dl}
\end{frame}

\begin{frame}
\begin{proof}
Nếu vector $x^*$ là nghiệm tối ưu của bài toán (F), ta có:
\begin{equation*}
\lambda^* = \frac{P(x^*)}{D(x^*)} \geq \frac{P(x)}{D(x)}, \: \forall x \in S_F
\end{equation*}
Tương tự,
\begin{equation} \label{pt2}
P(x)-\lambda^* D(x) \leq 0,  \forall x \in S_F
\end{equation}
Từ bất phương trình \eqref{pt2}, ta được:
\begin{equation*}
\Rightarrow \underset{x \in S_F}{\max}\{P(x)-\lambda^* D(x)\} = 0
\end{equation*}
Nếu vector $x^*$ là nghiệm tối ưu thì:
\begin{equation*}
P(x)-\lambda^*D(x) \leq P(x^*)-\lambda^* D(x^*) = 0, \: \forall x \in S_F
\end{equation*}
\end{proof}
\end{frame}

\begin{frame} \Large
Với $D(x)>0, \: \forall x \in S_F$, ta có:
\begin{equation}
\frac{\partial F(\lambda)}{\partial \lambda}= -D(x) < 0
\end{equation}
Đồng nghĩa $F(\lambda)$ giảm theo $\lambda$, từ đó ta thiết lập được thuật toán Dinkelbach.
\end{frame}

\begin{frame}{Thuật toán Dinkelbach}
\setlength{\parindent}{4em}
\noindent \textbf{Bước 1. Thiết lập} \\
Đặt $x^{(0)} \in S_F$, tính $\lambda^{(1)}:=\frac{P(x^{(0)})}{D(x^{(0)})}$, $k:=1$. \\
\noindent \textbf{Bước 2. Tìm nghiệm} \\
Tìm $x^{(k)}$ là nghiệm của bài toán $\underset{x \in S_F}{\max}\{P(x)-\lambda^{(k)} D(x)\}$ \\
\noindent \textbf{Bước 3. Kiểm tra} \\
Nếu $F(\lambda^{(k)})=0$ thì $x^*=x^{(k)}$ là nghiệm tối ưu và bài \\ toán được giải, nếu không chuyển sang bước 4. \\
\noindent \textbf{Bước 4. Cải thiện} \\
Đặt $\lambda^{(k+1)}:=\frac{P(x^{(k)})}{D(x^{(k)})}$, $k:=k+1$ và quay lại bước 2.
\end{frame}

\begin{frame}{Ví dụ minh hoạ}
\begin{equation} \large \label{F}
    \begin{split}
    (F) \quad Q(x) & = \frac{x_1+x_2+5}{3x_1+2x_2+15} \quad \longrightarrow Max \\
        & \left\{
        \begin{split}
        &3x_1 + x_2 \leq  6, \\
        &3x_1+4x_2 \leq 12 \\
        &x_1, x_2 \geq 0. \\
        \end{split}
        \right.    
    \end{split}
\end{equation}                
\end{frame}

\begin{frame}
Với $x=(0,0)^T$ ta có $x^{(0)} \in S_F$, cho $x^{(0)}=(0,0)^T$ ta có
\begin{equation*}
\lambda^{(1)}=\frac{P(x^{(0)})}{D(x^{(0)})}=\frac{1}{3}
\end{equation*}
Ta giải bài toán
\begin{equation*}
P(x)-\lambda^{(1)}D(x)=P(x)-\frac{1}{3}D(x)=\frac{1}{3}x_2 \rightarrow \max
\end{equation*}
\begin{equation*}
\Rightarrow x^{(1)}=(0,3)^T, F(\lambda^{(1)})=1
\end{equation*}
vì $F(\lambda^{(1)}) \neq 0$
\begin{equation*}
    \lambda^{(2)}=\frac{P(x^{(1)})}{D(x^{(1)})}=\frac{8}{21}
\end{equation*}
\begin{equation*}
    P(x)-\lambda^{(2)}D(x)=-\frac{1}{7}x_1+\frac{5}{21}x_2-\frac{5}{7} \rightarrow \max
\end{equation*}
\begin{equation*}
    \Rightarrow x^{(2)}=(0,3)^T, F(\lambda^{(2)})=0
\end{equation*}
Vậy nghiệm tối ưu của bài toán (F) là $x^*=(0,3)^T$ và giá trị tối ưu $Q(x^*)=\frac{8}{21}$.
\end{frame}
%DINKELBACH-end

%LD-start
\section{Thuật toán LandDoig - Dinkelbach}
\begin{frame}
   \center 
   \Huge Thuật toán LandDoig - Dinkelbach
\end{frame}



























\begin{frame}{Tối ưu phân tuyến tính nguyên hoàn toàn}
\begin{equation} \small
    \begin{split}
    (H) \quad Q(x) & = \frac{P(x)}{D(x)} \quad \longrightarrow Max \\
        & \left\{
        \begin{split}
        &Ax \leq  b, \\
        &x \geq 0, \text{nguyên} \\
        \end{split}
        \right.    
    \end{split}
\end{equation}            
\begin{itemize} \small
\item Bài toán $(H)$ gọi là bài toán \textbf{Tối ưu phân tuyến tính nguyên hoàn toàn.}
\item Trong đó $A$ là ma trận $m\times n$, $b=\begin{pmatrix}
    b_1 \\
    b_2 \\
    \vdots \\
    b_m
    \end{pmatrix}$, với $x\in \mathbb{Z}^n_+$. Tập $S_h:=\{x\in \mathbb{Z}^n_+: Ax\leq b\}$ là tập nghiệm của bài toán Tối ưu phân tuyến tính nguyên hoàn toàn. 
\item $P(x)=p^Tx+p_0$, với $p^T = (p_1 \: p_2 \: \ldots \: p_n)$ và $D(x)=d^Tx+d_0$, với $d^T = (d_1 \: d_2 \: \ldots \: d_n)$ ($D(x)>0, \forall x \in S_h$).
\end{itemize}
\end{frame}
\begin{frame}
\begin{figure}
    \begin{tikzpicture}[scale=1.2]
    \begin{axis}
        [
        xmin=0,xmax=80,
        ymin=0,ymax=80,
        xlabel={$x_1$},
        ylabel={$x_2$},
        grid=both,
        grid style={line width=.1pt, draw=darkgray!50},
        major grid style={line width=.2pt,draw=darkgray!50},
        axis lines=middle,
        yticklabel=\empty,
        xticklabel=\empty,
        enlargelimits={abs=0},
        samples=100,
        domain = -20:20,
        ]
        \filldraw[green!30, pattern=north west lines, pattern color=green!30, line width=1pt] (30, 25) -- (50, 20) -- (70, 30) -- (70, 50) -- (50, 65) -- (20, 50) -- cycle;
        \node[green] at (60,40) {\tiny $S_h$};
    \end{axis}
    \end{tikzpicture}  
    \caption{Tập nghiệm minh hoạ của bài toán (H)}
    \end{figure}
\end{frame}

\begin{frame}{Tối ưu phân tuyến tính nguyên bộ phận} \Large
\begin{equation}
    \begin{split}
    (B) \quad Q(x,y) & = \frac{P(x,y)}{D(x,y)} \quad \longrightarrow Max \\
        & \left\{
        \begin{split}
        &Ax + Gy \leq  b, \\
        &x \geq 0, \text{nguyên} \\
        &y \geq 0.
        \end{split}
        \right.    
    \end{split}
\end{equation}            
\end{frame}
\begin{frame}
\begin{itemize}
\item Bài toán $(B)$ gọi là bài toán \textbf{Tối ưu phân tuyến tính nguyên bộ phận.}
\item Trong đó $A$ là ma trận $m\times n$, $G$ là ma trận $m\times t$, $b=\begin{pmatrix}
    b_1 \\
    b_2 \\
    \vdots \\
    b_m
    \end{pmatrix}$, với $x\in \mathbb{Z}^n_+$ và $y \in \mathbb{R}^t_+$. Tập $S_b:=\{(x,y)\in \mathbb{Z}^n_+ \times \mathbb{R}^t_+: Ax+Gy\leq b\}$ là tập nghiệm của bài toán Tối ưu phân tuyến tính nguyên bộ phận. 
\item $P(x,y)=p^T_1x+p^T_2y+p_0$, với $p^T_1 = (p_1 \: p_2 \: \ldots \: p_n)$ và $p^T_2 = (p_1 \: p_2 \: \ldots \: p_t)$.
\item $D(x,y)=d^T_1x+d^T_2y+d_0$, với $d^T_1 = (d_1 \: d_2 \: \ldots \: d_n)$ và $d^T_2 = (d_1 \: d_2 \: \ldots \: d_t)$ ($D(x,y)>0, \forall x \in S_b$).
\end{itemize}
\end{frame}
\begin{frame}
\begin{figure}
    \begin{tikzpicture}[scale=1.2]
    \begin{axis}
        [
        xmin=0,xmax=80,
        ymin=0,ymax=80,
        xlabel={$x$},
        ylabel={$y$},
        grid style={line width=.1pt, draw=darkgray!50},
        major grid style={line width=.2pt,draw=darkgray!50},
        axis lines=middle,
        yticklabel=\empty,
        xticklabel=\empty,
        enlargelimits={abs=0},
        samples=100,
        xmajorgrids=true,
        domain = -20:20,
        ]
        \filldraw[green!30, pattern=north west lines, pattern color=green!30, line width=1pt] (30, 25) -- (50, 20) -- (70, 30) -- (70, 50) -- (50, 65) -- (20, 50) -- cycle;
        \node[green] at (60,40) {\tiny $S_b$};
    \end{axis}
    \end{tikzpicture}  
    \caption{Tập nghiệm minh hoạ của bài toán (B)}
    \end{figure}
\end{frame}























\begin{frame}
   \center 
   \Huge Mục tiêu phương pháp
\end{frame}

\begin{frame}{Bài toán quan tâm}
\begin{equation} \small \label{F}
    \begin{split}
    (F) \quad Q(x) & = \frac{P(x)}{D(x)} \quad \longrightarrow Max \\
        & \left\{
        \begin{split}
        &Ax \leq  b, \\
        &x \geq 0. \\
        \end{split}
        \right.    
    \end{split}
\end{equation}            
\begin{itemize} \small
\item Trong đó (F) là bài toán (H) với nghiệm thuộc tập số thực, ta nói (F) là bài toán mở rộng của (H).
\item Bài toán $(F)$ gọi là bài toán \textbf{Tối ưu phân tuyến tính thông thường} hay gọi đơn giản là bài toán Tối ưu phân tuyến tính (LFP relaxation).
\item Tập $S_F:=\{x\in \mathbb{R}^n_+: Ax\leq b\}$ là tập nghiệm của bài toán Tối ưu phân tuyến tính. 
\end{itemize}
\end{frame}

\begin{frame}{Phương pháp xử lý bài toán}
Giải bài toán (F) ta được nghiệm tối ưu ban đầu, ký hiệu vector $x_j \:\: (j=1,\ldots,n)$.
\begin{itemize}
\item Nếu $x_j \in \mathbb{Z}^n_+ \:\: (j=1,\ldots,n)$, thì bài toán (H) được giải.
\item Nếu $\exists x_j \notin \mathbb{Z}^n_+ \:\: (j=1,\ldots,n)$, ta thêm vào các ràng buộc 
\begin{equation*}
    x^i_j \leq \lfloor x_j \rfloor \text{ và } x^i_j \geq \lceil x_j \rceil 
\end{equation*}
vào bài toán (F) bằng \textsl{lý thuyết nhánh cận} và ta thiết lập được 2 bài toán con từ bài toán (F) ban đầu, ký hiệu $(F_1)$ và $(F_2)$:
\begin{equation*}
(F_1) \quad \underset{x \in S_1}{\max}Q_1(x),
\end{equation*} \\
và
\begin{equation*}
(F_2) \quad \underset{x \in S_2}{\max}Q_2(x).
\end{equation*} \\    
Lặp lại quá trình đến khi $\forall x_j \in \mathbb{Z}^n_+$.
\end{itemize}
\end{frame}
\begin{frame}
\begin{equation} \small \label{F}
    \begin{split}
    (F_1) \quad Q(x) & = \frac{P(x)}{D(x)} \quad \longrightarrow Max \\
        & \left\{
        \begin{split}
        &Ax \leq  b, \\
        &\color{red} x^i_j \leq \lfloor x_j \rfloor \\
        &x \geq 0. \\
        \end{split}
        \right.    
    \end{split}
\end{equation}
\begin{itemize}
    \item Tập $S_1:=S_F \cap \{ x: x^i_j \leq \lfloor x_j \rfloor \}$ là tập nghiệm tối ưu của bài toán con $(F_1)$.
    \end{itemize}               
\end{frame}

\begin{frame}
\begin{equation} \small \label{F}
    \begin{split}
    (F_1) \quad Q(x) & = \frac{P(x)}{D(x)} \quad \longrightarrow Max \\
        & \left\{
        \begin{split}
        &Ax \leq  b, \\
        &\color{red} x^i_j \geq \lceil x_j \rceil \\
        &x \geq 0. \\
        \end{split}
        \right.    
    \end{split}
\end{equation}
\begin{itemize}
    \item Tập $S_2:=S_F \cap \{ x: x^i_j \geq \lceil x_j \rceil \}$ là tập nghiệm tối ưu của bài toán con $(F_2)$.
\end{itemize}               
\end{frame}
    

\begin{frame}{Điều kiện nghiệm}
    \begin{itemize}
    \item Nếu tồn tại $(F_i)$ với $i=1,2$ không giải được $(S_i = \emptyset )$, ta gọi bài toán \textbf{vô nghiệm}.
    \item Giả sử $x^i$ là nghiệm tối ưu của bài toán $(F_i)$ và giá trị tối ưu là $Q_i(x)$ với $i = 1,2$.
    \begin{itemize}
    \item Nếu $\forall x^i \in Z^n_+$, ta nói $S_i$ là tập nghiệm thoả mãn bài toán tối ưu phần tuyến tính nguyên hoàn toàn, $Q^*_i(x)$ là giá trị tối ưu và bài toán con $(F_i)$ được giải \textbf{(gọt bởi nghiệm nguyên)}.
    \item Nếu $\exists x^i \notin Z^n_+$ đồng thời $Q_i(x) \leq Q^*_i(x)$, ta dừng phân nhánh và bỏ qua bài toán \textbf{(gọt bởi cận)}.
    \item Nếu $\exists x^i \notin Z^n_+$ đồng thời $Q_i(x) > Q^*_i(x)$, bài toán chưa tối ưu và có thể tiếp tục cải thiện.
    \iffalse
    Với $x^i_{j^'}$ là nghiệm không nguyên và phân thành 2 bài toán con $P_3$ với tập nghiệm $S_l := S_i \cap \{ (x,y): x_{j^{'}} \leq \lfloor x_{j^{'}} \rfloor \}$, $P_l$ với $S_l := S_i \cap \{ (x,y): x_{j^{'}} \geq \lceil x_{j^{'}} \rceil \}$ trong đó $l=3,4$ và lặp lại quá trình.
    \fi
    \end{itemize}
    \end{itemize}
\end{frame}

\begin{frame}{Ví dụ minh hoạ}

\end{frame}

\begin{frame}{Thuật toán LandDoig - Dinkelbach}
\begin{itemize}
\item Ta gọi bài toán $(F)$ có nút ban đầu là $N_0$, tương ứng mỗi bài toán tối ưu phân tuyến tính thông thường $(F_i)$ ứng với mỗi nút $N_i$ trên sơ đồ nhánh và $\mathcal{L}$ là danh sách chứa các nút được lập thông qua lý thuyết xác định cận và lý thuyết nghiệm.
\item Ta đánh dấu giá trị tối ưu tốt nhất và nghiệm tối ưu tốt nhất của bài toán lần lượt là $Q^*(x)$ và $x^*$.
\end{itemize}
\end{frame}

\begin{frame}{Thuật toán LandDoig - Dinkelbach}
\setlength{\parindent}{4em}
\noindent \textbf{Bước 0. Thiết lập} \\
Đặt $\mathcal{L}:={N_0}$. \\
Giải bài toán (F) được nghiệm $x^*=x$ và $Q^*(x)=Q(x)$. \\
\noindent \textbf{Bước 1. Kiểm tra} \\
Nếu $\forall x^i_j\in\mathbb{Z}^n_+$ thì vector $x$ là nghiệm của bài toán (H), \\ tương ứng $\mathcal{L}=\emptyset$ và bài toán được giải. \\
Nếu $\exists x^i_j\notin \mathbb{Z}^n_+$, tương ứng $\mathcal{L}\neq \emptyset$, ta chuyển sang bước 2. \\
\noindent \textbf{Bước 2. Phân nhánh} \\
Chọn nút $N_i$ ($x^i_j \notin \mathbb{Z}^n_+$ có phần thập phân lớn nhất) để \\ phân nhánh bằng cách thêm vào $S_k$ ràng buộc:
\begin{equation*}
x^i_j \leq \lfloor x_j \rfloor \text{ và } x^i_j \geq \lceil x_j \rceil 
\end{equation*} \\
lần lượt thành 2 bài toán con $S_i$ và $S_k$ ($k:=i+1$). \\
\end{frame}

\begin{frame}
\setlength{\parindent}{4em}
\noindent \textbf{Bước 3. Thiết lập bài toán con (các nút)} \\
Ta tập trung xử lý 2 bài toán con (các nút):
\begin{equation*}
(F_i) \quad \underset{x \in S_i}{\max}Q_i(x),
\end{equation*} \\
và
\begin{equation*}
(F_{k}) \quad \underset{x \in S_k}{\max}Q_k(x).
\end{equation*} \\
bằng cách dùng \textsl{thuật toán Dinkelbach}. \\
Nếu bài toán vô nghiệm hoặc $Q_i(x)\leq Q^*(x)$, đặt \\ $i:=k+1$ và quay lại bước 1, nếu không, chuyển sang \\ bước 4. \\
\end{frame}

\begin{frame}
\setlength{\parindent}{4em}
\noindent \textbf{Bước 4. Kiểm tra} \\
\end{frame}
%LD-end

\begin{frame}
\end{frame}

\end{document}